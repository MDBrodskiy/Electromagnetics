%%%%%%%%%%%%%%%%%%%%%%%%%%%%%%%%%%%%%%%%%%%%%%%%%%%%%%%%%%%%%%%%%%%%%%%%%%%%%%%%%%%%%%%%%%%%%%%%%%%%%%%%%%%%%%%%%%%%%%%%%%%%%%%%%%%%%%%%%%%%%%%%%%%%%%%%%%%%%%%%%%%
% Written By Michael Brodskiy
% Class: Fundamentals of Electromagnetics
% Professor: E. Marengo Fuentes
%%%%%%%%%%%%%%%%%%%%%%%%%%%%%%%%%%%%%%%%%%%%%%%%%%%%%%%%%%%%%%%%%%%%%%%%%%%%%%%%%%%%%%%%%%%%%%%%%%%%%%%%%%%%%%%%%%%%%%%%%%%%%%%%%%%%%%%%%%%%%%%%%%%%%%%%%%%%%%%%%%%

\documentclass[12pt]{article} 
\usepackage{alphalph}
\usepackage[utf8]{inputenc}
\usepackage[russian,english]{babel}
\usepackage{titling}
\usepackage{amsmath}
\usepackage{graphicx}
\usepackage{enumitem}
\usepackage{amssymb}
\usepackage[super]{nth}
\usepackage{everysel}
\usepackage{ragged2e}
\usepackage{geometry}
\usepackage{multicol}
\usepackage{fancyhdr}
\usepackage{cancel}
\usepackage{siunitx}
\usepackage{physics}
\usepackage{tikz}
\usepackage{mathdots}
\usepackage{yhmath}
\usepackage{cancel}
\usepackage{color}
\usepackage{array}
\usepackage{multirow}
\usepackage{gensymb}
\usepackage{tabularx}
\usepackage{extarrows}
\usepackage{booktabs}
\usepackage{lastpage}
\usepackage{float}
\usetikzlibrary{fadings}
\usetikzlibrary{patterns}
\usetikzlibrary{shadows.blur}
\usetikzlibrary{shapes}

\geometry{top=1.0in,bottom=1.0in,left=1.0in,right=1.0in}
\newcommand{\subtitle}[1]{%
  \posttitle{%
    \par\end{center}
    \begin{center}\large#1\end{center}
    \vskip0.5em}%

}
\usepackage{hyperref}
\hypersetup{
colorlinks=true,
linkcolor=blue,
filecolor=magenta,      
urlcolor=blue,
citecolor=blue,
}


\title{Final Exam}
\date{December 12, 2023}
\author{Michael Brodskiy\\ \small Professor: E. Marengo Fuentes}

\begin{document}

\maketitle

\begin{enumerate}

  \item 

    \begin{enumerate}

      \item 

        We know that, with an antenna centered at the origin, we may express the phasor as:

        $$\tilde{E}_\theta=j60I_o\left[ \frac{\cos\left( \frac{\pi}{2}\cos(\theta) \right)}{\sin(\theta)} \right]\left(\frac{e^{-jkR}}{R}\right)$$

      \item 

        We know that the resistance as perceive by the transmission line may be written as:

        $$R_{in}=R_{rad}+R_{loss}$$

        A half-wave dipole means that $l=.5\lambda$. We can use the following formula for radiation resistance:

        $$R_{rad}=80\pi^2\left( \frac{l}{\lambda} \right)^2$$

        And we can use the following formula for resistive loss:

        $$R_{loss}=$$

      \item 

      \item 

      \item 

      \item 

      \item 

      \item 

      \item 

    \end{enumerate}

  \item 

    \begin{enumerate}
        
      \item 

      \item 

      \item 

      \item 

      \item 

    \end{enumerate}

  \item

    \begin{enumerate}

      \item 

        We can find the parallel polarization reflection coefficient using the formula:

        $$\Gamma_{\parallel}=\frac{\eta_2\cos(\theta_t)-\eta_1\cos(\theta_i)}{\eta_2\cos(\theta_i)+\eta_1\cos(\theta_i)}$$

        We can first find the transmitted angle using Snell's Law:

        $$n_1\sin(\theta_i)=n_2\sin(\theta_t)$$
        $$\theta_t=\sin^{-1}\left(\frac{n_1\sin(\theta_i)}{n_1}\right)$$
        $$\theta_t=\sin^{-1}\left(\frac{(1)\sin(50)}{\sqrt{4}}\right)$$
        $$\theta_t=22.52$$

        We apply this information to our formula:

        $$\Gamma_{\parallel}=\frac{\sqrt{4}\cos(22.52)-(1)\cos(50)}{\sqrt{4}\cos(22.52)+(1)\cos(50)}$$
        $$\boxed{\Gamma_{\parallel}=.4838}$$


      \item 

        We know that the transmission coefficient is related to the reflection coefficient for parallel polarization by:

        $$\tau_{\parallel}=(1+\Gamma_{\parallel})\frac{\cos(\theta_i)}{\cos(\theta_t)}$$

        This gives us:

        $$\tau_{\parallel}=(1+.4838)\frac{\cos(50)}{\cos(22.52)}$$
        $$\boxed{\tau_{\parallel}=1.0325}$$

      \item 

        For perpendicular polarization, we know that the reflection coefficient may be expressed as:

        $$\Gamma_{\perp}=\frac{\eta_2\cos(\theta_i)-\eta_1\cos(\theta_t)}{\eta_2\cos(\theta_i)+\eta_1\cos(\theta_t)}$$

        This yields:

        $$\Gamma_{\perp}=\frac{2\cos(50)-(1)\cos(22.52)}{2\cos(50)+(1)\cos(22.52)}$$
        $$\boxed{\Gamma_{\perp}=.1638}$$

      \item 

        We know that the transmission coefficient relation to the perpendicular reflection coefficient is:

        $$\tau_{\perp}=(1+\Gamma_{\perp})$$

        This gives us:

        $$\tau_{\perp}=(1+.1638)$$
        $$\boxed{\tau_{\perp}=1.1638}$$

      \item 

        The Brewster angle can be determined using:

        $$\theta_B=\tan^{-1}\left( \sqrt{\frac{\varepsilon_2}{\varepsilon_1}} \right)$$

        This gives us:

        $$\theta_{B\parallel}=\tan^{-1}\left( \sqrt{4} \right)$$
        $$\theta_{B\parallel}=\tan^{-1}\left( 2\right)$$
        $$\boxed{\theta_{B\parallel}=63.435^{\circ}}$$

      \item 

        We know that, for a nonmagnetic material (as the one described in the problem), the Brewster angle exists only for the parallel polarization (thus the use of $\theta_{B\parallel}$). A beam entering a medium at a this angle of incidence would fully transmit its perpendicular component (only the perpendicular component is reflected). This can be explained in the diagram below:

        \begin{figure}[h]
          \centering
          \tikzset{every picture/.style={line width=0.75pt}} %set default line width to 0.75pt        

\begin{tikzpicture}[x=0.75pt,y=0.75pt,yscale=-1,xscale=1]
%uncomment if require: \path (0,504); %set diagram left start at 0, and has height of 504

%Shape: Rectangle [id:dp24828463924463207] 
\draw  [color={rgb, 255:red, 80; green, 227; blue, 194 }  ,draw opacity=1 ][fill={rgb, 255:red, 80; green, 227; blue, 194 }  ,fill opacity=1 ] (138,229) -- (455,229) -- (455,333) -- (138,333) -- cycle ;
%Straight Lines [id:da47522454577861106] 
\draw    (171.5,128.79) -- (270.09,227.38) ;
\draw [shift={(271.5,228.79)}, rotate = 225] [color={rgb, 255:red, 0; green, 0; blue, 0 }  ][line width=0.75]    (10.93,-3.29) .. controls (6.95,-1.4) and (3.31,-0.3) .. (0,0) .. controls (3.31,0.3) and (6.95,1.4) .. (10.93,3.29)   ;
%Straight Lines [id:da1053230153406346] 
\draw    (271.5,228.79) -- (370.09,327.38) ;
\draw [shift={(371.5,328.79)}, rotate = 225] [color={rgb, 255:red, 0; green, 0; blue, 0 }  ][line width=0.75]    (10.93,-3.29) .. controls (6.95,-1.4) and (3.31,-0.3) .. (0,0) .. controls (3.31,0.3) and (6.95,1.4) .. (10.93,3.29)   ;
%Straight Lines [id:da4504340662665598] 
\draw    (271.5,228.79) -- (370.09,130.2) ;
\draw [shift={(371.5,128.79)}, rotate = 135] [color={rgb, 255:red, 0; green, 0; blue, 0 }  ][line width=0.75]    (10.93,-3.29) .. controls (6.95,-1.4) and (3.31,-0.3) .. (0,0) .. controls (3.31,0.3) and (6.95,1.4) .. (10.93,3.29)   ;

% Text Node
\draw (140,225.6) node [anchor=south west] [inner sep=0.75pt]    {$n_{1}$};
% Text Node
\draw (140,232.4) node [anchor=north west][inner sep=0.75pt]    {$n_{2}$};
% Text Node
\draw (169.5,125.39) node [anchor=south east] [inner sep=0.75pt]    {$( \perp ,\parallel )$};
% Text Node
\draw (373.5,125.39) node [anchor=south west] [inner sep=0.75pt]    {$( \Gamma_{\perp}\perp,0)$};
% Text Node
\draw (373.5,325.39) node [anchor=south west] [inner sep=0.75pt]    {$( \tau_{\perp}\perp , \parallel )$};


\end{tikzpicture}

          \caption{Brewster Angle Scenario}
          \label{fig:1}
        \end{figure}

        Given a coordinate set $(\perp,\parallel)$ representing each components strength, we can see that for the incident wave and transmitted wave, the perpendicular components are equal, since it is fully transmitted. On the other hand, there is some parallel reflection, which can be represented by the transmission and reflection coefficients.

      \item 

        We now assume that the same scenario occurs, but with the Brewster angle. This gives us:

        $$n_1\sin(\theta_i)=n_2\sin(\theta_t)$$
        $$\theta_t=\sin^{-1}\left(\frac{n_1\sin(\theta_i)}{n_2}\right)$$
        $$\theta_t=\sin^{-1}\left(\frac{\sin(63.435)}{2}\right)$$
        $$\boxed{\theta_t=26.565^{\circ}}$$

        From here, we can find the perpendicular reflection coefficient:

        $$\Gamma_{\perp}=\frac{\eta_2\cos(\theta_i)-\eta_1\cos(\theta_t)}{\eta_2\cos(\theta_i)+\eta_1\cos(\theta_t)}$$
        $$\Gamma_{\perp}=\frac{2\cos(63.435)-\cos(26.565)}{2\cos(63.435)+\cos(26.565)}$$
        $$\boxed{\Gamma_{\perp}=0}$$

        Note that the above result makes sense, as we expect the perpendicular component to be fully transmitted. The parallel reflection becomes:

        $$\Gamma_{\parallel}=\frac{\eta_2\cos(\theta_t)-\eta_1\cos(\theta_i)}{\eta_2\cos(\theta_t)+\eta_1\cos(\theta_i)}$$
        $$\Gamma_{\parallel}=\frac{2\cos(26.565)-\cos(63.435)}{2\cos(26.565)+\cos(63.435)}$$
        $$\boxed{\Gamma_{\parallel}=.6}$$

        The perpendicular transmission may be expressed as:

        $$\tau_{\perp}=(1+\Gamma_{\perp})$$
        $$\boxed{\tau_{\perp}=1}$$

        Again, this makes sense as the perpendicular component will be fully transmitted. The parallel transmission is:

        $$\tau_{\parallel}=(1+\Gamma_{\parallel})\frac{\cos(\theta_i)}{\cos(\theta_t)}$$
        $$\tau_{\parallel}=(1+.6)\frac{\cos(63.435)}{\cos(26.565)}$$
        $$\boxed{\tau_{\parallel}=.8}$$

      \item 

    \end{enumerate}

  \item

    \begin{enumerate}

      \item 

        Given by the fact that field $\tilde{E}_i$ is directed in the $\bold{\hat{y}}$ direction, while being influenced by $x$ and $z$ components, we know that the wave is affected by \underline{perpendicular} polarization.

      \item 

        Given $3x+4z$, we may write:

        $$\theta_i=\tan^{-1}\left( \frac{3}{4} \right)$$
        $$\boxed{\theta_i=36.87^{\circ}}$$

      \item 

        This may be written in the time domain as:

        $$\tilde{E}_i=\bold{\hat{y}}20e^{-j(3x+4z)}$$
        $$E_i=20\cos(\omega t-(3x+4z))\bold{\hat{y}}$$

        We can find the angular frequency using the formula:

        $$\omega=ck$$

        This gives us:

        $$\omega=c\sqrt{3^2+4^2}$$
        $$\omega=5c$$
        $$\omega=1.5\cdot10^{9}\left[ \frac{\text{rad}}{\si{\second}} \right]$$

        Thus, we get:

        $$\boxed{E_i=20\cos((1.5\cdot10^9) t-(3x+4z))\bold{\hat{y}}}$$

      \item 

        The average power density is defined by the formula:

        $$S_{avg}=\frac{|E|^2}{2\eta}$$

        Since the incident wave is traveling in air, we may write:

        $$\eta=\eta_o=376.819[\si{\ohm}]$$

        Since, from the time domain, we know the magnitude of the wave, we may write:

        $$S_{avg}=\frac{(20)^2}{2\cdot376.819}$$
        $$\boxed{S_{avg}=.531\left[ \frac{\si{\watt}}{\si{\meter\squared}} \right]}$$

      \item 

        Since the wave is perpendicularly polarized, know that the magnitude of the reflected wave can be defined using:

        $$|E^r|=\Gamma|E^i|$$

        The reflection coefficient may be defined as:

        $$\Gamma_{\perp}=\frac{\eta_2\cos(\theta_i)-\eta_1\cos(\theta_t)}{\eta_2\cos(\theta_i)+\eta_1\cos(\theta_t)}$$

        Now, we need to find the transmitted angle:

        $$\theta_t=\sin^{-1}\left( \frac{n_1\sin(\theta_i)}{n_2} \right)$$
        $$\theta_t=\sin^{-1}\left( \frac{\sin(36.87)}{2} \right)$$
        $$\theta_t=17.48^{\circ}$$

        From here, we return to the reflection coefficient:

        $$\Gamma_{\perp}=\frac{2\cos(36.87)-\cos(17.48)}{2\cos(36.87)+\cos(17.48)}$$
        $$\Gamma_{\perp}=.253$$

        This means that the magnitude becomes:

        $$|E^r|=.253\cdot20=5.06$$

        This gives us:

        $$S_{avg}=\frac{|5.06|^2}{2\cdot376.819}$$
        $$\boxed{S_{avg}=.03398\left[ \frac{\si{\watt}}{\si{\meter\squared}} \right]}$$

    \end{enumerate}

\end{enumerate}

\end{document}

