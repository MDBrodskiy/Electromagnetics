%%%%%%%%%%%%%%%%%%%%%%%%%%%%%%%%%%%%%%%%%%%%%%%%%%%%%%%%%%%%%%%%%%%%%%%%%%%%%%%%%%%%%%%%%%%%%%%%%%%%%%%%%%%%%%%%%%%%%%%%%%%%%%%%%%%%%%%%%%%%%%%%%%%%%%%%%%%%%%%%%%%
% Written By Michael Brodskiy
% Class: Fundamentals of Electromagnetics
% Professor: E. Marengo Fuentes
%%%%%%%%%%%%%%%%%%%%%%%%%%%%%%%%%%%%%%%%%%%%%%%%%%%%%%%%%%%%%%%%%%%%%%%%%%%%%%%%%%%%%%%%%%%%%%%%%%%%%%%%%%%%%%%%%%%%%%%%%%%%%%%%%%%%%%%%%%%%%%%%%%%%%%%%%%%%%%%%%%%

\documentclass[12pt]{article} 
\usepackage{alphalph}
\usepackage[utf8]{inputenc}
\usepackage[russian,english]{babel}
\usepackage{titling}
\usepackage{amsmath}
\usepackage{graphicx}
\usepackage{enumitem}
\usepackage{amssymb}
\usepackage[super]{nth}
\usepackage{everysel}
\usepackage{ragged2e}
\usepackage{geometry}
\usepackage{multicol}
\usepackage{fancyhdr}
\usepackage{cancel}
\usepackage{siunitx}
\usepackage{physics}
\usepackage{tikz}
\usepackage{mathdots}
\usepackage{yhmath}
\usepackage{cancel}
\usepackage{color}
\usepackage{array}
\usepackage{multirow}
\usepackage{gensymb}
\usepackage{tabularx}
\usepackage{extarrows}
\usepackage{booktabs}
\usepackage{lastpage}
\usepackage{float}
\usetikzlibrary{fadings}
\usetikzlibrary{patterns}
\usetikzlibrary{shadows.blur}
\usetikzlibrary{shapes}

\geometry{top=1.0in,bottom=1.0in,left=1.0in,right=1.0in}
\newcommand{\subtitle}[1]{%
  \posttitle{%
    \par\end{center}
    \begin{center}\large#1\end{center}
    \vskip0.5em}%

}
\usepackage{hyperref}
\hypersetup{
colorlinks=true,
linkcolor=blue,
filecolor=magenta,      
urlcolor=blue,
citecolor=blue,
}


\title{Exam 2}
\date{\today}
\author{Michael Brodskiy\\ \small Professor: E. Marengo Fuentes}

\begin{document}

\maketitle

\begin{enumerate}

  \item

    \begin{enumerate}

      \item 

      \item 

      \item 

      \item 

      \item 

      \item 

      \item 

      \item 

      \item 

      \item 

    \end{enumerate}

  \item

    \begin{enumerate}

      \item 

      \item 

      \item 

      \item 

    \end{enumerate}

  \item

        First and foremost, we know $\sin(x)=\cos\left( x-\frac{\pi}{2} \right)$ and $-\cos(x)=\cos(x-\pi)$. This will be important for the following problems.

    \begin{enumerate}

      \item 

        We are given the equation:

        $$\vec{E}=\bold{\hat{x}}5\sin(\omega t + z)-\bold{\hat{y}}5\cos(\omega t + z)\,\left[ \frac{\si{\volt}}{\si{\meter}} \right]$$

        This can be rewritten as:

        $$\vec{E}=\bold{\hat{x}}5\cos\left(\omega t + z - \frac{\pi}{2}\right)+\bold{\hat{y}}5\cos\left(\omega t + z-\pi\right)\,\left[ \frac{\si{\volt}}{\si{\meter}} \right]$$

        We can see that $E_y$ lags $E_x$ by $\pi/2$, or $90^{\circ}$. This, in tandem with the equal magnitudes define the above equation as \underline{Right-Hand Polarized}.

      \item 

        We are given the equation:

        $$\vec{E}=\bold{\hat{z}}\cos(\omega t + x)-\bold{\hat{y}}\sin(\omega t + x)\,\left[ \frac{\si{\volt}}{\si{\meter}} \right]$$

        This can be rewritten as:

        $$\vec{E}=\bold{\hat{z}}\cos\left(\omega t + x\right)+\bold{\hat{y}}\cos\left(\omega t + z-\frac{3\pi}{2}\right)\,\left[ \frac{\si{\volt}}{\si{\meter}} \right]$$
        $$\vec{E}=\bold{\hat{z}}\cos\left(\omega t + x\right)+\bold{\hat{y}}\cos\left(\omega t + z+\frac{\pi}{2}\right)\,\left[ \frac{\si{\volt}}{\si{\meter}} \right]$$

        We can see that $E_y$ leads $E_z$ by $\pi/2$, or $90^{\circ}$. This, in tandem with the equal magnitudes define the above equation as \underline{Right-Hand Polarized}.

      \item 

        We are given the equation:

        $$\vec{H}=\bold{\hat{x}}5\cos(\omega t - z)+\bold{\hat{y}}5\sin(\omega t -z)\,\left[ \frac{\si{\milli\ampere}}{\si{\meter}} \right]$$

        This can be rewritten as:

        $$\vec{H}=\bold{\hat{x}}5\cos(\omega t - z)+\bold{\hat{y}}5\cos\left(\omega t -z-\frac{\pi}{2}\right)\,\left[ \frac{\si{\milli\ampere}}{\si{\meter}} \right]$$

        We can see that $H_y$ lags $H_x$ by $\pi/2$, or $90^{\circ}$. This, in tandem with the equal magnitudes define the above equation as \underline{Right-Hand Polarized}.

    \end{enumerate}

\end{enumerate}

\end{document}

