%%%%%%%%%%%%%%%%%%%%%%%%%%%%%%%%%%%%%%%%%%%%%%%%%%%%%%%%%%%%%%%%%%%%%%%%%%%%%%%%%%%%%%%%%%%%%%%%%%%%%%%%%%%%%%%%%%%%%%%%%%%%%%%%%%%%%%%%%%%%%%%%%%%%%%%%%%%%%%%%%%%
% Written By Michael Brodskiy
% Class: Fundamentals of Electromagnetics
% Professor: E. Marengo Fuentes
%%%%%%%%%%%%%%%%%%%%%%%%%%%%%%%%%%%%%%%%%%%%%%%%%%%%%%%%%%%%%%%%%%%%%%%%%%%%%%%%%%%%%%%%%%%%%%%%%%%%%%%%%%%%%%%%%%%%%%%%%%%%%%%%%%%%%%%%%%%%%%%%%%%%%%%%%%%%%%%%%%%

\documentclass[12pt]{article} 
\usepackage{alphalph}
\usepackage[utf8]{inputenc}
\usepackage[russian,english]{babel}
\usepackage{titling}
\usepackage{amsmath}
\usepackage{graphicx}
\usepackage{enumitem}
\usepackage{amssymb}
\usepackage[super]{nth}
\usepackage{everysel}
\usepackage{ragged2e}
\usepackage{geometry}
\usepackage{multicol}
\usepackage{fancyhdr}
\usepackage{cancel}
\usepackage{siunitx}
\usepackage{physics}
\usepackage{tikz}
\usepackage{mathdots}
\usepackage{yhmath}
\usepackage{cancel}
\usepackage{color}
\usepackage{array}
\usepackage{multirow}
\usepackage{gensymb}
\usepackage{tabularx}
\usepackage{extarrows}
\usepackage{booktabs}
\usepackage{lastpage}
\usepackage{float}
\usetikzlibrary{fadings}
\usetikzlibrary{patterns}
\usetikzlibrary{shadows.blur}
\usetikzlibrary{shapes}

\geometry{top=1.0in,bottom=1.0in,left=1.0in,right=1.0in}
\newcommand{\subtitle}[1]{%
  \posttitle{%
    \par\end{center}
    \begin{center}\large#1\end{center}
    \vskip0.5em}%

}
\usepackage{hyperref}
\hypersetup{
colorlinks=true,
linkcolor=blue,
filecolor=magenta,      
urlcolor=blue,
citecolor=blue,
}


\title{Exam 2}
\date{\today}
\author{Michael Brodskiy\\ \small Professor: E. Marengo Fuentes}

\begin{document}

\maketitle

\begin{enumerate}

  \item

    \begin{enumerate}

      \item We can tell from the given function that the direction of \underline{propagation is $-\bold{\hat{x}}$}

      \item The frequency is equal to the coefficient of $t$ divided by $2\pi$:

        $$\boxed{f=\frac{1}{2\pi}(5\pi\cdot10^7)=2.5\cdot10^7\left[ \si{\hertz} \right]}$$

      \item The wavelength may be calculated using the following formula:

        $$\lambda=\frac{u_p}{f}$$

        First, we calculate the propagation speed:

        $$u_p=\frac{1}{\sqrt{\mu\varepsilon}}=\frac{1}{\sqrt{(36)(4\pi\cdot10^{-7})(8.85\cdot10^{-12})}}=5\cdot10^{7}\left[ \frac{\si{\meter}}{\si{\second}} \right]$$

        And then the wavelength:

        $$\boxed{\lambda=\frac{5\cdot10^7}{2.5\cdot10^7}=2[\si{\meter}]}$$

      \item The wave number $k$ can be defined in several ways, one of which is:

        $$k=\frac{2\pi}{\lambda}$$

        This gives us:

        $$\boxed{k=\frac{2\pi}{2}=\pi\left[ \frac{\text{rad}}{\si{\meter}} \right]}$$

      \item The medium is lossless because there is no imaginary part to the permittivity or permeability

      \item The wave impedance, $\eta$, may be defined as:

        $$\eta=\sqrt{\frac{\mu}{\varepsilon}}$$

        Thus we can find:

        $$\boxed{\eta=\sqrt{\frac{(4\pi\cdot10^{-7})(4)}{(8.85\cdot10^{-12})(9)}}=251.213[\si{\ohm}]}$$

      \item As a phasor, we can express the electric field as:

        $$\boxed{\tilde{E}=10e^{j\pi x}\bold{\hat{y}}-3e^{j\left(\pi x-\frac{\pi}{2}\right)}\bold{\hat{z}}\,\left[ \frac{\si{\volt}}{\si{\meter}} \right]}$$

      \item The average power density can be found according to the Poynting vector:

        $$S_{avg}=\bold{\hat{x}}\frac{|\tilde{E}|^2}{2\eta}$$

        We can find the magnitude of $\tilde{E}$:

        $$|\tilde{E}|=\sqrt{(10)^2+(-3)^2}=10.44$$

        Using the formula, we get:

        $$\boxed{S_{avg}=\bold{\hat{x}}\frac{10.44^2}{502.416}=.216939\,\left[ \frac{\si{\watt}}{\si{\meter\squared}} \right]}$$

      \item Now, we can find the magnetic field by using:

        $$\tilde{H}=\frac{1}{\eta}(\bold{\hat{x}}\times\tilde{E})$$

        Substituting the values we know, we get:

        $$\tilde{H}=\frac{1}{251.213}\left[ \bold{\hat{x}}\times\left( 10e^{j\pi x}\bold{\hat{y}}-3e^{\pi x-\frac{\pi}{2}}\bold{\hat{z}} \right) \right]$$
        $$\tilde{H}=\frac{1}{251.213}\left( 10e^{j\pi x}\bold{\hat{z}}+3e^{\pi x-\frac{\pi}{2}}\bold{\hat{y}} \right)$$
        $$\boxed{\tilde{H}=.0398e^{j\pi x}\bold{\hat{z}}+.01194e^{j\left(\pi x-\frac{\pi}{2}\right)}\bold{\hat{y}}\,\left[ \frac{\si{\ampere}}{\si{\meter}} \right]}$$

      \item We can then convert the above to a time-domain function:

        $$\boxed{H=.0398\cos( \left( 5\pi\cdot10^7 \right)t+\pi x)\bold{\hat{z}}+.01194\sin( \left( 5\pi\cdot10^7 \right)t+\pi x)\bold{\hat{y}}\,\left[ \frac{\si{\ampere}}{\si{\meter}} \right]}$$

    \end{enumerate}

  \item

    We are given the equation:

    $$H(0,t)=\bold{\hat{y}}50\sin(\omega t+23^{\circ})\,\left[ \frac{\si{\milli\ampere}}{\si{\meter}} \right]$$

    \begin{enumerate}

      \item 

        We know that, for a field equation, two terms go to zero when $z\to0$. These are:

        $$\text{The attenuation factor: }e^{-\alpha z}$$
        $$\text{The phase value: }-\beta z$$

        We first check whether this is a good conductor:

        $$\frac{\sigma}{\omega}=\frac{3}{2\pi\cdot10^4}=.000048$$
        $$\varepsilon=(70)(8.85\cdot10^{-12})=6.195\cdot10^{-10}$$

        Since the ratio of conductance to angular frequency is much greater than the permittivity, this is a good conductor. We can solve for these using the formulas:

        $$\alpha=\sqrt{\pi f\mu\sigma}$$
        $$\beta=\alpha$$

        This gives us:

        $$\alpha=\sqrt{\pi(10^4)(4\pi\cdot10^{-7})(3)}=.344144,\left[ \frac{\text{Np}}{\si{\meter}} \right]$$
        $$\beta=.344144\,\left[ \frac{\text{rad}}{\si{\meter}} \right]$$

        The phase shift may also be converted to radians:

        $$23\cdot\frac{2\pi}{360}=.401426$$

        Thus, we can fully define the formula as:

        $$H(z,t)=50e^{-.344144z}\sin( \left( 2\pi\cdot10^4\right)t-.344144z+.401426)\bold{\hat{y}}\,\left[ \frac{\si{\milli\ampere}}{\si{\meter}} \right]$$
        $$\boxed{H(z,t)=50e^{-.344144z}\cos( \left( 2\pi\cdot10^4\right)t-.344144z-1.16937)\bold{\hat{y}}\,\left[ \frac{\si{\milli\ampere}}{\si{\meter}} \right]}$$

      \item 

        We know by definition, that:

        $$E=\eta (\bold{\hat{z}}\times H)$$

        Where:

        $$\eta=(1+j)\frac{\alpha}{\sigma}$$

        We can calculate the intrinsic impedance as:

        $$\eta=(1+j)\frac{.344144}{3}=.114715+.114715j$$

        Multiplying $H(z,t)$ by $\eta$ gives us:

        $$E(z,t)=(.114715+.114715j)50e^{-.344144z}\cos( \left( 2\pi\cdot10^4\right)t-.344144z-1.16937)$$
        $$E(z,t)=8.11158e^{j\frac{\pi}{4}}e^{-.344144z}\cos( \left( 2\pi\cdot10^4\right)t-.344144z-1.16937)$$
        $$E(z,t)=8.11158e^{-.344144z}\cos( \left( 2\pi\cdot10^4\right)t-.344144z-.383972)$$
        $$\boxed{E(z,t)=8.11158e^{-.344144z}\cos( \left( 2\pi\cdot10^4\right)t-.344144z-.383972)\bold{\hat{x}}\,\left[ \frac{\si{\milli\volt}}{\si{\meter}} \right]}$$

      \item 

        To find the average power density at depth $z$, we may use the Poynting vector. This can be defined as:

        $$S_{avg}=\bold{\hat{z}}\frac{|\tilde{E}(0)|^2}{2|\eta|}e^{-2\alpha z}\cos(\theta)$$

        We know the magnitude of $\eta$ and $\tilde{E}$, as well as the angle and $\alpha$, which gives us:

        $$S_{avg}=\bold{\hat{z}}\frac{|.00811158|^2}{2|.16223|}e^{-2(.344144) z}\cos\left( \frac{\pi}{4} \right)$$

        This gives:

        $$\boxed{S_{avg}=.143e^{-.688288z}\bold{\hat{z}}\,\left[ \frac{\si{\milli\watt}}{\si{\meter\squared}} \right]}$$

      \item 

        The power drop off can be found using:

        $$10^{-\frac{20}{10}}\cdot100=1\%$$

        Which we can then use in:

        $$.01=e^{-.688z}$$
        $$z=\frac{\ln(.01)}{-.688}$$
        $$\boxed{z=6.69356[\si{\meter}]}$$

    \end{enumerate}

  \item

        First and foremost, we know $\sin(x)=\cos\left( x-\frac{\pi}{2} \right)$ and $-\cos(x)=\cos(x-\pi)$. This will be important for the following problems.

    \begin{enumerate}

      \item 

        We are given the equation:

        $$\vec{E}=\bold{\hat{x}}5\sin(\omega t + z)-\bold{\hat{y}}5\cos(\omega t + z)\,\left[ \frac{\si{\volt}}{\si{\meter}} \right]$$

        This can be rewritten as:

        $$\vec{E}=\bold{\hat{x}}5\cos\left(\omega t + z - \frac{\pi}{2}\right)+\bold{\hat{y}}5\cos\left(\omega t + z-\pi\right)\,\left[ \frac{\si{\volt}}{\si{\meter}} \right]$$

        We can see that $E_y$ lags $E_x$ by $\pi/2$, or $90^{\circ}$. This, in tandem with the equal magnitudes define the above equation as \underline{Right-Hand Polarized}.

      \item 

        We are given the equation:

        $$\vec{E}=\bold{\hat{z}}\cos(\omega t + x)-\bold{\hat{y}}\sin(\omega t + x)\,\left[ \frac{\si{\volt}}{\si{\meter}} \right]$$

        This can be rewritten as:

        $$\vec{E}=\bold{\hat{z}}\cos\left(\omega t + x\right)+\bold{\hat{y}}\cos\left(\omega t + z-\frac{3\pi}{2}\right)\,\left[ \frac{\si{\volt}}{\si{\meter}} \right]$$
        $$\vec{E}=\bold{\hat{z}}\cos\left(\omega t + x\right)+\bold{\hat{y}}\cos\left(\omega t + z+\frac{\pi}{2}\right)\,\left[ \frac{\si{\volt}}{\si{\meter}} \right]$$

        We can see that $E_y$ leads $E_z$ by $\pi/2$, or $90^{\circ}$. This, in tandem with the equal magnitudes define the above equation as \underline{Right-Hand Polarized}.

      \item 

        We are given the equation:

        $$\vec{H}=\bold{\hat{x}}5\cos(\omega t - z)+\bold{\hat{y}}5\sin(\omega t -z)\,\left[ \frac{\si{\milli\ampere}}{\si{\meter}} \right]$$

        This can be rewritten as:

        $$\vec{H}=\bold{\hat{x}}5\cos(\omega t - z)+\bold{\hat{y}}5\cos\left(\omega t -z-\frac{\pi}{2}\right)\,\left[ \frac{\si{\milli\ampere}}{\si{\meter}} \right]$$

        We can see that $H_y$ lags $H_x$ by $\pi/2$, or $90^{\circ}$. This, in tandem with the equal magnitudes define the above equation as \underline{Right-Hand Polarized}.

    \end{enumerate}

\end{enumerate}

\end{document}

