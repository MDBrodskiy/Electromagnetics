%%%%%%%%%%%%%%%%%%%%%%%%%%%%%%%%%%%%%%%%%%%%%%%%%%%%%%%%%%%%%%%%%%%%%%%%%%%%%%%%%%%%%%%%%%%%%%%%%%%%%%%%%%%%%%%%%%%%%%%%%%%%%%%%%%%%%%%%%%%%%%%%%%%%%%%%%%%%%%%%%%%
% Written By Michael Brodskiy
% Class: Fundamentals of Electromagnetics
% Professor: E. Marengo Fuentes
%%%%%%%%%%%%%%%%%%%%%%%%%%%%%%%%%%%%%%%%%%%%%%%%%%%%%%%%%%%%%%%%%%%%%%%%%%%%%%%%%%%%%%%%%%%%%%%%%%%%%%%%%%%%%%%%%%%%%%%%%%%%%%%%%%%%%%%%%%%%%%%%%%%%%%%%%%%%%%%%%%%

\include{Includes.tex}

\title{Fundamentals of Electromagnetics}
\date{\today}
\author{Michael Brodskiy\\ \small Professor: E. Marengo Fuentes}

\begin{document}

\maketitle

\begin{itemize}

  \item There are two kinds of fields, electric ($E$) and magnetic ($H$)

  \item There are two kinds of fluxes, electric ($D$) and magnetic ($B$)

    $$\boxed{D=\varepsilon E}$$
    $$\boxed{B=\mu H}$$

    \begin{itemize}

      \item Where $\varepsilon$ is the permittivity and $\mu$ is the permeability

    \end{itemize}

  \item The electromagnetic force:

    $$F=\underbrace{qE}_\text{Electric}+\underbrace{qv\times B}_\text{Magnetic }$$

    \begin{itemize}

      \item The magnetic component is referred to as Lorentz's Force

    \end{itemize}

  \item Connection to gravitation

    \begin{itemize}

      \item The electric force from particle 1 to 2 may be written as:

        $$F=\frac{q_1q_2}{4\pi\varepsilon R^2}\hat{r}_{12}$$

      \item From particle 2 to 1, it may be written as

        $$F=\frac{q_1q_2}{4\pi\varepsilon R^2}\hat{r}_{21}=\frac{q_1q_2}{4\pi\varepsilon R^2}(-\hat{r}_{12})$$

      \item This is known as the Coulomb force

      \item Combining this with force equation above, we see:

        $$E_{q_1}=\frac{q_1}{4\pi\varepsilon R^2}\hat{r}_{12}$$

      \item Gravitation between two masses can be written as:

        $$F_{1\to2}=\frac{m_1m_2}{R^2}G\hat{r}_{12}$$

    \end{itemize}

  \item Signals and Systems

    \begin{itemize}

      \item Systems take in some input (such as a signal), $x$, use a rule $\mathcal{L}$, and map it to an output, $y$

      \item Linear Systems — Obey the principle of superposition; that is, the output of a sum of inputs = the sum of outputs of the individual inputs

          $$\mathcal{L}\Sigma=\Sigma\mathcal{L}$$

          \begin{itemize}

            \item Linear systems can be split up into two further groups: time-variant and time-invariant (LTI)

          \end{itemize}

    \end{itemize}

  \item Fourier/Frequency Domain Representation

    \begin{itemize}

      \item F.T.\ of $x(t)=\tilde{x}(\omega)$

      \item F.T.\ of $y(t)=\tilde{y}(\omega)$

      \item This makes:

        $$\tilde{y}(\omega)=H(\omega)\tilde{x}(\omega)$$

        where $H(\omega)$ is the transfer function

      \item We can recall from circuits:

        $$y(t)=V_0\cos(\omega t+\phi)$$

      \item With Euler representation, we get:

        $$e^{j\theta)=\cos(\theta)+j\sin(\theta)$$
        $$\tilde{Y}(\omega)=\text{ phasor }=V_0e^{j\theta}$$

      \item By definition, the Fourier transform is:

        $$f(t)=\frac{1}{2\pi}\int_{-\infty}^{\infty}\tilde{F}(\omega)e^{j\omega t}\,d\omega\text{, if signal $f(t)\in$ real:}$$
        $$=Re\int_0^{\infty}\,d\omega\,e^{j\omega t}\tilde{A}(\omega)$$
        $$=\int_0^{\infty}\,d\omega\,Re\left( e^{j\omega t} \tilde{A}(\omega)\right)$$

      \item This is called the analytic signal representation

    \end{itemize}

  \item Waves

    \begin{itemize}

      \item A wave can be thought of as a propagating disturbance in a medium

      \item There is finite propagation time

      \item The most general way to describe a propagating, unidimensional wave traveling in the positive $x$ direction can be written as follows:

        $$f(x,t)=F\left(t-\frac{x}{c}\right)$$

        \begin{itemize}

          \item Special case: cosine

            $$F(t)=A\cos(\omega t+\phi)$$
            $$f(x,t)=A\cos\left(\omega t + \phi - \frac{x}{c}\right)\footnote{This wave is in the $+x$ direction}$$

        \end{itemize}

      \item The most general wave traveling in the $-x$ direction:

        $$f(x,t)=F\left( t+\frac{x}{c} \right)$$

      \item We know $\omega=2\pi f$

      \item This yields us $\dfrac{\omega}{c}=k$, or the wave number (of a wave)

      \item Thus,

        $$\omega t-\frac{x}{c}=\omega \left( t-k\frac{x}{\omega} \right)=\omega t-kx$$

      \item Finally, we obtain:

        $$f(x,t)=A\cos(\omega t+\phi - kx)$$

        This is called the canonical form of a wave traveling in the positive $x$ direction.

    \end{itemize}

  \item Traveling Waves

    \begin{itemize}

      \item Wave maintains shape, no distortion

      \item Medium is nondispersive (properties do not depend on frequency)

      \item For a sinusoidal, $T$ is the period; for a delayed sinusoid, $T\cong$ pulse length

      \item Given a pulse defined by $f(t-d/c)$:

      \item If $d/c<<T$, the delayed version $\approx$ original

        \begin{itemize}

          \item Limited delay, same signal at input and at output, we can ignore the finite propagation speed of wave, circuit theory applies

        \end{itemize}

      \item If $d/c$ is NOT $<<T$

        \begin{itemize}

          \item We need electromagnetics, circuit theory does NOT apply

        \end{itemize}

      \item If $d<<\lambda$ (where $\lambda$ is wavelength), circuit theory applies

      \item $f(t-x/c)=y(t)\rightarrow f(t)=A\cos(\omega t+\phi)\rightarrow y(t)=A\cos(\omega t+\phi-kx)$

        \begin{itemize}

          \item $k=\frac{2\pi}{\lambda}$ is the wave number

          \item $\lambda=\frac{c}{f}$

          \item Propagates in $+x$

          \item $A$ is the amplitude, $\phi$ is the phase, $\omega$ is the frequency, $f$ is the frequency in hertz, $2\pi f$ is the frequenzy in rad/s, and $\lambda$ is the wavelength

        \end{itemize}

      \item An exponential term adds attenuation: $f(t-x/c)=f(t)e^{-\alpha x}$

        \begin{itemize}
            
          \item $\alpha$ is the attentuation factor

          \item $y(x,t)=Ae^{-\alpha x}\cos(\omega t -kx+\phi$ for the $+x$ direction

          \item $y(x,t)=Ae^{\alpha x}\cos(\omega t +kx+\phi$ for the $-x$ direction

          \item Phasor: $\tilde{y}(x)=\underbrace{\left(Ae^{-\alpha x}\right)}_{\text{attenuation}}\,\underbrace{\left(e^{-jkx}\right)}_{\text{phase form}}\left(e^{j\phi}\right)$

        \end{itemize}

    \end{itemize}

\end{itemize}

\end{document}

