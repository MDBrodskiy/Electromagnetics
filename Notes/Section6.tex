%%%%%%%%%%%%%%%%%%%%%%%%%%%%%%%%%%%%%%%%%%%%%%%%%%%%%%%%%%%%%%%%%%%%%%%%%%%%%%%%%%%%%%%%%%%%%%%%%%%%%%%%%%%%%%%%%%%%%%%%%%%%%%%%%%%%%%%%%%%%%%%%%%%%%%%%%%%%%%%%%%%
% Written By Michael Brodskiy
% Class: Fundamentals of Electromagnetics
% Professor: E. Marengo Fuentes
%%%%%%%%%%%%%%%%%%%%%%%%%%%%%%%%%%%%%%%%%%%%%%%%%%%%%%%%%%%%%%%%%%%%%%%%%%%%%%%%%%%%%%%%%%%%%%%%%%%%%%%%%%%%%%%%%%%%%%%%%%%%%%%%%%%%%%%%%%%%%%%%%%%%%%%%%%%%%%%%%%%

\documentclass[12pt]{article} 
\usepackage{alphalph}
\usepackage[utf8]{inputenc}
\usepackage[russian,english]{babel}
\usepackage{titling}
\usepackage{amsmath}
\usepackage{graphicx}
\usepackage{enumitem}
\usepackage{amssymb}
\usepackage[super]{nth}
\usepackage{everysel}
\usepackage{ragged2e}
\usepackage{geometry}
\usepackage{multicol}
\usepackage{fancyhdr}
\usepackage{cancel}
\usepackage{siunitx}
\usepackage{physics}
\usepackage{tikz}
\usepackage{mathdots}
\usepackage{yhmath}
\usepackage{cancel}
\usepackage{color}
\usepackage{array}
\usepackage{multirow}
\usepackage{gensymb}
\usepackage{tabularx}
\usepackage{extarrows}
\usepackage{booktabs}
\usepackage{lastpage}
\usepackage{float}
\usetikzlibrary{fadings}
\usetikzlibrary{patterns}
\usetikzlibrary{shadows.blur}
\usetikzlibrary{shapes}

\geometry{top=1.0in,bottom=1.0in,left=1.0in,right=1.0in}
\newcommand{\subtitle}[1]{%
  \posttitle{%
    \par\end{center}
    \begin{center}\large#1\end{center}
    \vskip0.5em}%

}
\usepackage{hyperref}
\hypersetup{
colorlinks=true,
linkcolor=blue,
filecolor=magenta,      
urlcolor=blue,
citecolor=blue,
}


\title{Wave Reflection and Transmission}
\date{\today}
\author{Michael Brodskiy\\ \small Professor: E. Marengo Fuentes}

\begin{document}

\maketitle

\begin{itemize}

  \item Given a phasor wave $\tilde{E}$ and a perpendicular propagation vector $\hat{k}$, we may write:

    $$\tilde{E}=E_o\hat{u}e^{-jk\hat{k}\cdotr}$$
    $$\hat{u}\cdot\hat{k}=0$$
    $$r=(x,y,z)$$

  \item The propagation constant may be defined as:

    $$c_{prop}=\frac{\omega}{c}$$

    where 

    $$c=\frac{c_o}{\sqrt{\varepsilon_r\mu_r}}$$

  \item We will apply transmission line theory to normal incidence and oblique incidence

    \begin{itemize}

      \item The reflection coefficient, $\Gamma$, will be defined relative to the electric field

      \item There is also a transmission coefficient ($\tau$)
        
      \item We can write our formulas as:

        $$\tilde{E}_t=E_o\tau\bold{\hat{x}}e^{-jk_zz}$$
        $$\tilde{H}_t=\frac{E_o\tau}{\eta_2}\bold{\hat{y}}e^{-jk_zz}$$

      \item We can then obtain the reflections

        $$\tilde{E}_r=-\bold{\hat{x}}E_o\Gamma e^{jk_1z}$$
        $$\tilde{H}_r=-\bold{\hat{y}}\frac{E_o\Gamma}{\eta_1} e^{jk_1z}$$

      \item We can also get the formula

        $$1+\Gamma=\tau$$
        $$\Gamma=\frac{\eta_2-\eta_1}{\eta_2+\eta_1}$$
        $$\tau=\frac{2\eta_2}{\eta_2+\eta_1}$$

      \item The Poynting Vector can be redefined as

        $$S_{av}=\frac{\bold{\hat{z}}|E_o|^2}{2\eta_1}\left( 1-|\Gamma|^2 \right)$$

    \end{itemize}

  \item Oblique Incidence

    \begin{itemize}

      \item Snell's Laws apply

      \item The index of refraction is:

        $$c=\frac{c_o}{n}$$
        $$n=\sqrt{\varepsilon_r}$$
        $$n_1\sin(\theta_i)=n_2\sin(\theta_t)$$

    \end{itemize}

\end{itemize}

\end{document}

