%%%%%%%%%%%%%%%%%%%%%%%%%%%%%%%%%%%%%%%%%%%%%%%%%%%%%%%%%%%%%%%%%%%%%%%%%%%%%%%%%%%%%%%%%%%%%%%%%%%%%%%%%%%%%%%%%%%%%%%%%%%%%%%%%%%%%%%%%%%%%%%%%%%%%%%%%%%%%%%%%%%
% Written By Michael Brodskiy
% Class: Fundamentals of Electromagnetics
% Professor: E. Marengo Fuentes
%%%%%%%%%%%%%%%%%%%%%%%%%%%%%%%%%%%%%%%%%%%%%%%%%%%%%%%%%%%%%%%%%%%%%%%%%%%%%%%%%%%%%%%%%%%%%%%%%%%%%%%%%%%%%%%%%%%%%%%%%%%%%%%%%%%%%%%%%%%%%%%%%%%%%%%%%%%%%%%%%%%

\include{Includes.tex}

\title{Transmission Lines}
\date{\today}
\author{Michael Brodskiy\\ \small Professor: E. Marengo Fuentes}

\begin{document}

\maketitle

\begin{itemize}

  \item Transmission lines connect inputs to loads

    \begin{itemize}

      \item $l$ is the length of the transmission lines

      \item If $l$ is not much smaller than $\lambda$, we need detailed analysis

      \item If $l$ is comparable to $\lambda$, then we can not use the lumped parameter model

      \item We can, however, partition the transmission lines into segments where $l<<\lambda$, then we can apply Kirchoff's circuit laws to each subdivided segment

      \item For an imperfect dielectric, there is some loss

      \item Some per unit-length properties:

        \begin{enumerate}

          \item Resistance per unit length: $R'$ (ohm per meter)

            $$\boxed{R=R'\Delta z}$$

          \item Inductance per unit length: $L'$ (Henry per meter)

            $$\boxed{L=L'\Delta z}$$

          \item Capacitance per unit length: $C'$ (Farad per meter)

            $$\boxed{C=C'\Delta z}$$

          \item Conductance per unit length: $G'$ (Siemens per meter)

            $$\boxed{G=G'\Delta z}$$

        \end{enumerate}

    \end{itemize}

  \item Using this, we obtain the Helmholtz Equation:

    $$\boxed{\frac{d^2 \tilde{V}(z)}{dz^2}-\gamma^2\tilde{V}(z)=0}$$

    \begin{itemize}

      \item Where $\gamma=\sqrt{(j\omega C' + G')(j\omega L' + R')}$

    \end{itemize}

\end{itemize}

\end{document}

