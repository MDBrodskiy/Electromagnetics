%%%%%%%%%%%%%%%%%%%%%%%%%%%%%%%%%%%%%%%%%%%%%%%%%%%%%%%%%%%%%%%%%%%%%%%%%%%%%%%%%%%%%%%%%%%%%%%%%%%%%%%%%%%%%%%%%%%%%%%%%%%%%%%%%%%%%%%%%%%%%%%%%%%%%%%%%%%%%%%%%%%
% Written By Michael Brodskiy
% Class: Fundamentals of Electromagnetics
% Professor: E. Marengo Fuentes
%%%%%%%%%%%%%%%%%%%%%%%%%%%%%%%%%%%%%%%%%%%%%%%%%%%%%%%%%%%%%%%%%%%%%%%%%%%%%%%%%%%%%%%%%%%%%%%%%%%%%%%%%%%%%%%%%%%%%%%%%%%%%%%%%%%%%%%%%%%%%%%%%%%%%%%%%%%%%%%%%%%

\include{Includes.tex}

\title{Transmission Lines}
\date{\today}
\author{Michael Brodskiy\\ \small Professor: E. Marengo Fuentes}

\begin{document}

\maketitle

\begin{itemize}

  \item Transmission lines connect inputs to loads

    \begin{itemize}

      \item $l$ is the length of the transmission lines

      \item If $l$ is not much smaller than $\lambda$, we need detailed analysis

      \item If $l$ is comparable to $\lambda$, then we can not use the lumped parameter model

      \item We can, however, partition the transmission lines into segments where $l<<\lambda$, then we can apply Kirchoff's circuit laws to each subdivided segment

      \item For an imperfect dielectric, there is some loss

      \item Some per unit-length properties:

        \begin{enumerate}

          \item Resistance per unit length: $R'$ (ohm per meter)

            $$\boxed{R=R'\Delta z}$$

          \item Inductance per unit length: $L'$ (Henry per meter)

            $$\boxed{L=L'\Delta z}$$

          \item Capacitance per unit length: $C'$ (Farad per meter)

            $$\boxed{C=C'\Delta z}$$

          \item Conductance per unit length: $G'$ (Siemens per meter)

            $$\boxed{G=G'\Delta z}$$

        \end{enumerate}

    \end{itemize}

  \item Using this, we obtain the Helmholtz Equation:

    $$\boxed{\frac{d^2 \tilde{V}(z)}{dz^2}-\gamma^2\tilde{V}(z)=0}$$

    \begin{itemize}

      \item Where $\gamma=\sqrt{(j\omega C' + G')(j\omega L' + R')}$

    \end{itemize}

  \item For a Unidimensional Wave Equation:

    \begin{itemize}

      \item The characteristic equation becomes: $m^2-\gamma^2=0\rightarrow m=\pm\gamma$

      \item The solutions to the above differential equation are superpositions of $\{e^{\gamma z},e^{-\gamma z}\}$

      \item $\gamma=\sqrt{(R'+j\omega L')(G'+j\omega C')}$, where:

        \begin{itemize}

          \item $R'$ is the resistance per unit length

          \item $L'$ is the impedance per unit length

          \item $j=\sqrt{-1}$

          \item $\gamma=R_e\gamma + I_m\gamma$

          \item $R_e\gamma=\alpha$, the attenuation constant

          \item $I_m\gamma=\beta$, the phase constant (sometimes, notation used is $k=\beta$, like the wave number)

        \end{itemize}

      $$\widetilde{V}(z)=V_o^+e^{-\gamma z}+V_o^-e^{\gamma z}$$

    \item We know from the definitions above that $\gamma=\alpha + j\beta$

    \item For $\alpha>0$, we are dealing with a (passive) lossy material ($\alpha=0$) is a loss less material

    \item For $\alpha<0$, we are dealing with a gainy material

    \item The wave coming from the source is known as the ``incident'' wave, and the wave combing from the load is known as ``reflected''

      $$-\frac{dV}{dz}=(R'+j\omega L')\widetilde{I}(z)$$

    \item Solving this by incorporation the equation for $\widetilde{V}$ above, we obtain:

      $$\widetilde{I}=\left( V_o^+e^{-\gamma z}-V_o^-e^{\gamma z} \right)\left( \frac{\gamma}{R'+j\omega L'} \right)$$

    \item We then define $z_o=\displaystyle \frac{R'+j\omega L}{\gamma}$ as our characteristic impedance:

      $$\widetilde{I}=\frac{1}{z_o}\left( V_o^+e^{-\gamma z}-V_o^-e^{\gamma z} \right)$$

    \item There are thus two unknowns: $V_o^+$, which depends on source, and $V_o^-$, which is the reflected wave amplitude (depends on load)

      \begin{itemize}

        \item The incident wave (in phase domain): $\widetilde{V}=V_o^+e^{-\alpha z}e^{-j\beta z}$

        \item In time domain, this becomes: $v(z,t)=|V_o^+|e^{-\alpha z}\cos(\omega t-\beta z+\phi_+)$\footnote{$\beta$ represents the wave number}

      \end{itemize}

    \item For a lossless line ($R'=0,G'=0$, and a pure inductor and capacitor)

      $$\gamma=\sqrt{(jC'\omega)(jL'\omega)}\rightarrow\beta=\omega\sqrt{L'C'}=\frac{\omega}{U_{ph}}$$
      $$U_{ph}=\frac{1}{\sqrt{L'C'}}=\frac{1}{\sqrt{\mu\varepsilon}}=c\text{ in dielectric}$$

    \item The reflection coefficient ($\Gamma$) is given by:

      $$\Gamma=\frac{V_o^-}{V_o^+}$$
      $$V_{load}=V_o^+(1+\Gamma)$$
      $$I_{load}=\frac{V_o^+}{z_o}(1-\Gamma)$$

    \item The normalized load impedance: $\hat{z_L}=\frac{z_L}{z_o}$

      $$\Gamma=\frac{z_l-z_o}{z_l+z_o}\rightarrow\Gamma = \frac{\hat{z_L}-1}{\hat{z_L}+1}$$

    \item Special Cases:

      \begin{itemize}

        \item Short Circuit:

          $$\Gamma_{sc}=-1$$

        \item Open Circuit:

          $$\Gamma_{oc}=1$$

          \begin{itemize}

            \item Reactive load, no real absorption

          \end{itemize}

      \end{itemize}

    \item A Phase-Shifted $\Gamma$ would look as follows:

      $$\Gamma_l=\left(\frac{z_L-z_o}{z_L+z_o}\right)e^{-j(2\beta l)}$$

    \end{itemize}

  \item Standing Waves

    \begin{itemize}

      \item $\tilde{V}(d)=V_o^+e^{-j\beta d}+V_o^+\Gamma e^{j\beta d}$

        $$|\tilde{V}(d)|=|V_o^+\left[ 1+|\Gamma|^2 + 2|\Gamma|\cos(2\beta d-\theta_r) \right]^{\frac{1}{2}}$$

        \begin{itemize}

          \item This fluctuates between a minimum ($V_{min}$) and maximum ($V_{max}$) value

          \item Maximum/minimum possible value is determined by the $\cos$ term and is $\pm2|\Gamma$ 

        \end{itemize}

      \item From the formula for $\Gamma$, we see that $\Gamma=0$ when the load impedance matches the internal impedance

        \begin{itemize}

          \item This is known as load matching

        \end{itemize}

      \item Thus, when $\Gamma=0$, we know $|\tilde{V}(d)|=|V_0^+|$

      \item The standing wave ratio is defined as follows:

        $$SWR=\frac{1+|\Gamma|}{1-|\Gamma|}$$

      \item This measures the matching level of load to the line; ideally, $\Gamma=0$, and $SWR=1$; worst case, $|\Gamma|=1$, and $SWR=\infty$

      \item In reference to the maximums and minimums, these occur when $2\beta d-\theta_r=2n\pi$ and $2\beta d-\theta_r=(2n+1)\pi$, respectively. This results in:

        $$\left\{\begin{array}{l} d_{max}=\displaystyle\frac{2n\pi+\theta_r}{2\beta}\\\\d_{min}=\displaystyle\frac{(2n+1)\pi+\theta_r}{\2\beta}\end{array}$$

    \end{itemize}

  \item Power Flow in Transmission Line

    \begin{itemize}

      \item Average Power

        $$\langle P_i\rangle=\frac{|V_o|^2}{2Z_o}[\si{\watt}]$$

    \end{itemize}

\end{itemize}

\end{document}

