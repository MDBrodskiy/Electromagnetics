%%%%%%%%%%%%%%%%%%%%%%%%%%%%%%%%%%%%%%%%%%%%%%%%%%%%%%%%%%%%%%%%%%%%%%%%%%%%%%%%%%%%%%%%%%%%%%%%%%%%%%%%%%%%%%%%%%%%%%%%%%%%%%%%%%%%%%%%%%%%%%%%%%%%%%%%%%%%%%%%%%%
% Written By Michael Brodskiy
% Class: Fundamentals of Electromagnetics
% Professor: E. Marengo Fuentes
%%%%%%%%%%%%%%%%%%%%%%%%%%%%%%%%%%%%%%%%%%%%%%%%%%%%%%%%%%%%%%%%%%%%%%%%%%%%%%%%%%%%%%%%%%%%%%%%%%%%%%%%%%%%%%%%%%%%%%%%%%%%%%%%%%%%%%%%%%%%%%%%%%%%%%%%%%%%%%%%%%%

\documentclass[12pt]{article} 
\usepackage{alphalph}
\usepackage[utf8]{inputenc}
\usepackage[russian,english]{babel}
\usepackage{titling}
\usepackage{amsmath}
\usepackage{graphicx}
\usepackage{enumitem}
\usepackage{amssymb}
\usepackage[super]{nth}
\usepackage{everysel}
\usepackage{ragged2e}
\usepackage{geometry}
\usepackage{multicol}
\usepackage{fancyhdr}
\usepackage{cancel}
\usepackage{siunitx}
\usepackage{physics}
\usepackage{tikz}
\usepackage{mathdots}
\usepackage{yhmath}
\usepackage{cancel}
\usepackage{color}
\usepackage{array}
\usepackage{multirow}
\usepackage{gensymb}
\usepackage{tabularx}
\usepackage{extarrows}
\usepackage{booktabs}
\usepackage{lastpage}
\usepackage{float}
\usetikzlibrary{fadings}
\usetikzlibrary{patterns}
\usetikzlibrary{shadows.blur}
\usetikzlibrary{shapes}

\geometry{top=1.0in,bottom=1.0in,left=1.0in,right=1.0in}
\newcommand{\subtitle}[1]{%
  \posttitle{%
    \par\end{center}
    \begin{center}\large#1\end{center}
    \vskip0.5em}%

}
\usepackage{hyperref}
\hypersetup{
colorlinks=true,
linkcolor=blue,
filecolor=magenta,      
urlcolor=blue,
citecolor=blue,
}


\title{Plane-Wave Propagation}
\date{\today}
\author{Michael Brodskiy\\ \small Professor: E. Marengo Fuentes}

\begin{document}

\maketitle

\begin{itemize}

  \item We will have problems about fields from sources radiating

    \begin{itemize}

      \item In a source-free region, there are no electric and magnetic fields. That would make Maxwell's equations:

        $$\nabla\cdot\tilde{E}=0$$
        $$\nabla\times\tilde{E}=-j\omega\mu\tilde{H}$$
        $$\nabla\cdot\tilde{H}=0$$
        $$\nabla\times\tilde{H}=\tilde{J}+j\omega\varepsilon\tilde{E}$$

      \item In a region with a source, Maxwell's equations may be written as:

        $$\nabla\cdot\tilde{E}=\frac{\rho_V}{\varepsilon}$$
        $$\nabla\times\tilde{E}=-j\omega\mu\tilde{H}$$
        $$\nabla\cdot\tilde{H}=0$$
        $$\nabla\times\tilde{H}=\tilde{J}_{cond}+j\omega\varepsilon\tilde{E}$$

      \item There are two components that contribute to current density:

        $$\tilde{J}=\tilde{J}_{impressed}+\tilde{J}_{cond}$$

        \begin{itemize}

          \item Impressed is from a source, and conductive is an intrinsic property

            $$\tilde{J}=\sigma\tilde{E}$$

        \end{itemize}
        
      \item The homogenous form of Maxwell's equations can thus be written as:

        $$\nabla\cdot\tilde{E}=0$$
        $$\nabla\times\tilde{E}=-j\omega\mu\tilde{H}$$
        $$\nabla\cdot\tilde{H}=0$$
        $$\nabla\times\tilde{H}=(\sigma+j\omega\varepsilon)\tilde{E}$$

    \end{itemize}

  \item Far from sources, fields propagate like a plane wave (the circle becomes so large, it can be approximated as a line)

  \item The $\tilde{J}$ conduction component is known as drift

    \begin{itemize}

      \item If the conductivity is non-zero, we can see from the equation above that:

        $$\nabla\times\tilde{H}=j\omega\left( \varepsilon-\frac{j\sigma}{\omega} \right)\tilde{E}$$
        $$\varepsilon_c=\varepsilon-\frac{j\sigma}{\omega}$$

        \begin{itemize}

          \item If $\frac{\sigma}{\omega}<<\varepsilon$, then the material is an insulator, and:

            $$\varepsilon_c=\varepsilon$$

          \item If $\frac{\sigma}{\omega}>>\varepsilon$, the material is conductive; Note: this means that a good conductor depends on the angular frequency

          \item We can determine that:

            $$\tan(\theta)=\frac{\sigma}{\omega\varepsilon}$$

          \item If the tangent is approximately 0, the conductivity is negligible

        \end{itemize}

    \end{itemize}

  \item Wave Equations

    \begin{itemize}

      \item From manipulating the first non-zero equation, we get:

        $$\nabla\times\nabla\times\vec{E}-\omega^2\mu\varepsilon_c\tilde{E}=0$$

      \item We can also get:

        $$\nabla^2\vec{E}-\omega\mu\varepsilon_c\tilde{E}=0$$

      \item In lossy media, $\sigma\neq0$ and $\varepsilon_c$ is a complex value. We assign $\gamma=-\omega^2\mu\varepsilon_c$, and can now write:

        $$(\nabla^2-\gamma^2)\tilde{E}=0$$
        $$(\nabla^2-\gamma^2)\tilde{H}=0$$

      \item For lossless media, $\sigma=0$, and $\varepsilon_c$ is purely real

      \item We can rewrite the equation as:

        $$(\nabla^2+k^2)\tilde{E}(r)=0,\text{ where }r=(x,y,z)$$

        \begin{itemize}

          \item $k$ is the wave number, $\omega\sqrt{\mu\varepsilon}$

        \end{itemize}

    \end{itemize}

  \item Mapping from source to field is known as radiation

    \begin{itemize}

      \item A long distance from the source, the spherical wavefront may be approximated to a line

    \end{itemize}

  \item Using one of our equations from above, and plugging in the value of $\nabla\times\tilde{H}$, we can obtain:

    $$\nabla^2\tilde{E}+\omega^2\mu\varepsilon\tilde{E}-j\omega\mu\sigma\tilde{E}=0$$

  \item For plane waves, we end up with two important equations:

    $$\tilde{H}=\frac{1}{\eta}\hat{k}\times\tilde{E}$$
    $$\tilde{E}=-\eta\hat{k}\times\tilde{E}$$

    \begin{itemize}

      \item Where $\tilde{E}$ and $\tilde{H}$ are plane waves, $\hat{k}$ is the direction of propagation, and $\eta=\sqrt{\frac{\mu}{\varepsilon}}$. This means we know:

        $$\tilde{E}\cdot\hat{k}=0$$
        $$\tilde{H}\cdot\hat{k}=0$$

    \end{itemize}

  \item Polarization

    \begin{itemize}

      \item Occurs when two components or more components of a wave are in different phases

    \end{itemize}

\end{itemize}

\end{document}

