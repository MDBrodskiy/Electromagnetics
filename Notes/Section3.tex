%%%%%%%%%%%%%%%%%%%%%%%%%%%%%%%%%%%%%%%%%%%%%%%%%%%%%%%%%%%%%%%%%%%%%%%%%%%%%%%%%%%%%%%%%%%%%%%%%%%%%%%%%%%%%%%%%%%%%%%%%%%%%%%%%%%%%%%%%%%%%%%%%%%%%%%%%%%%%%%%%%%
% Written By Michael Brodskiy
% Class: Fundamentals of Electromagnetics
% Professor: E. Marengo Fuentes
%%%%%%%%%%%%%%%%%%%%%%%%%%%%%%%%%%%%%%%%%%%%%%%%%%%%%%%%%%%%%%%%%%%%%%%%%%%%%%%%%%%%%%%%%%%%%%%%%%%%%%%%%%%%%%%%%%%%%%%%%%%%%%%%%%%%%%%%%%%%%%%%%%%%%%%%%%%%%%%%%%%

\documentclass[12pt]{article} 
\usepackage{alphalph}
\usepackage[utf8]{inputenc}
\usepackage[russian,english]{babel}
\usepackage{titling}
\usepackage{amsmath}
\usepackage{graphicx}
\usepackage{enumitem}
\usepackage{amssymb}
\usepackage[super]{nth}
\usepackage{everysel}
\usepackage{ragged2e}
\usepackage{geometry}
\usepackage{multicol}
\usepackage{fancyhdr}
\usepackage{cancel}
\usepackage{siunitx}
\usepackage{physics}
\usepackage{tikz}
\usepackage{mathdots}
\usepackage{yhmath}
\usepackage{cancel}
\usepackage{color}
\usepackage{array}
\usepackage{multirow}
\usepackage{gensymb}
\usepackage{tabularx}
\usepackage{extarrows}
\usepackage{booktabs}
\usepackage{lastpage}
\usepackage{float}
\usetikzlibrary{fadings}
\usetikzlibrary{patterns}
\usetikzlibrary{shadows.blur}
\usetikzlibrary{shapes}

\geometry{top=1.0in,bottom=1.0in,left=1.0in,right=1.0in}
\newcommand{\subtitle}[1]{%
  \posttitle{%
    \par\end{center}
    \begin{center}\large#1\end{center}
    \vskip0.5em}%

}
\usepackage{hyperref}
\hypersetup{
colorlinks=true,
linkcolor=blue,
filecolor=magenta,      
urlcolor=blue,
citecolor=blue,
}


\title{Electrostatics, Boundary Conditions, Potentials, \& More}
\date{\today}
\author{Michael Brodskiy\\ \small Professor: E. Marengo Fuentes}

\begin{document}

\maketitle

\begin{itemize}

  \item We begin with Maxwell's Equations:

    $$\nabla\cdot D=P\quad\text{Gauss's Law}$$

    \begin{itemize}

      \item Where $D$ is the electric flux density

    $$\nabla\times E=-\frac{\partial}{\partial t}B\quad\text{Faraday's Law}$$

    $$\nabla\cdot B=0\quad\text{Magnetic Gauss's Law}$$

    $$\nabla\times H=J+\frac{\partial}{\partial t}D\quad\text{Amp\'ere's Law}$$

      \item $\rho$ is the charge density in coulombs per cubic meter, $J$ is the current density, in amp\'eres per square meter

    \end{itemize}

  \item In electrostatics, these laws mean:

    $$\frac{\partial}{\partial t}=0$$
    $$\nabla\cdot D=\rho$$
    $$\nabla\times E=0$$

  \item In magnetostatics, these laws mean:

    $$\frac{\partial}{\partial t}=0$$
    $$\nabla\cdot B=0$$
    $$\nabla\times H=J$$

  \item The emphasis in electrodynamics is:

    $$\frac{\partial}{\partial t}\neq0$$

  \item Combining equations, we get:

    $$J=\rho v$$

    where $v$ is the velocity of the charge in meters per second

  \item Work

    \begin{itemize}

      \item Moving a particle from point 1 to 2 would take work:

        $$W=\int_1^2 -Eq\,dl$$

      \item It can be expressed as a difference in potential energies:

        $$V_2-V_1=\int_1^2 -Eq\,dl$$

      \item Moving a charge from infinity:

        $$V_2-V_{\infty}=\int_\infty^2 -Eq\,dl$$
        $$V=-\int E\,dl\longleftrightarrow E=-\nabla V$$

      \item The Laplacian operator can be expressed as:

        $$\nabla^2=\nabla\cdot\nabla$$
        $$\nabla^2(V)=-\frac{\rho}{\varepsilon}$$

    \end{itemize}

  \item Source Point Analysis

    \begin{itemize}

      \item The voltage with respect to a source point ($r'$) and an observation point ($r$) can be expressed as:

        $$V(r)=\int\frac{\rho(r')\,dr'}{4\pi\varepsilon|r-r'|}$$

      \item We can also learn:

        $$\nabla^2G(r,r')=\delta(r-r')$$

        where

        $$G(r,r')=-\frac{1}{4\pi\varepsilon|r-r'|}$$

      \item This is known as the Green's function or impulse response

    \end{itemize}

\end{itemize}

\end{document}

