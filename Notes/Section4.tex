%%%%%%%%%%%%%%%%%%%%%%%%%%%%%%%%%%%%%%%%%%%%%%%%%%%%%%%%%%%%%%%%%%%%%%%%%%%%%%%%%%%%%%%%%%%%%%%%%%%%%%%%%%%%%%%%%%%%%%%%%%%%%%%%%%%%%%%%%%%%%%%%%%%%%%%%%%%%%%%%%%%
% Written By Michael Brodskiy
% Class: Fundamentals of Electromagnetics
% Professor: E. Marengo Fuentes
%%%%%%%%%%%%%%%%%%%%%%%%%%%%%%%%%%%%%%%%%%%%%%%%%%%%%%%%%%%%%%%%%%%%%%%%%%%%%%%%%%%%%%%%%%%%%%%%%%%%%%%%%%%%%%%%%%%%%%%%%%%%%%%%%%%%%%%%%%%%%%%%%%%%%%%%%%%%%%%%%%%

\documentclass[12pt]{article} 
\usepackage{alphalph}
\usepackage[utf8]{inputenc}
\usepackage[russian,english]{babel}
\usepackage{titling}
\usepackage{amsmath}
\usepackage{graphicx}
\usepackage{enumitem}
\usepackage{amssymb}
\usepackage[super]{nth}
\usepackage{everysel}
\usepackage{ragged2e}
\usepackage{geometry}
\usepackage{multicol}
\usepackage{fancyhdr}
\usepackage{cancel}
\usepackage{siunitx}
\usepackage{physics}
\usepackage{tikz}
\usepackage{mathdots}
\usepackage{yhmath}
\usepackage{cancel}
\usepackage{color}
\usepackage{array}
\usepackage{multirow}
\usepackage{gensymb}
\usepackage{tabularx}
\usepackage{extarrows}
\usepackage{booktabs}
\usepackage{lastpage}
\usepackage{float}
\usetikzlibrary{fadings}
\usetikzlibrary{patterns}
\usetikzlibrary{shadows.blur}
\usetikzlibrary{shapes}

\geometry{top=1.0in,bottom=1.0in,left=1.0in,right=1.0in}
\newcommand{\subtitle}[1]{%
  \posttitle{%
    \par\end{center}
    \begin{center}\large#1\end{center}
    \vskip0.5em}%

}
\usepackage{hyperref}
\hypersetup{
colorlinks=true,
linkcolor=blue,
filecolor=magenta,      
urlcolor=blue,
citecolor=blue,
}


\title{Magnetostatics}
\date{\today}
\author{Michael Brodskiy\\ \small Professor: E. Marengo Fuentes}

\begin{document}

\maketitle

\begin{itemize}

  \item From the prior lecture, we know:

    $$\left\{\begin{array}{l}\nabla\cdot D=\rho\\\nabla\times E=-\frac{\partial}{\partial t}B\end{array}$$

  \item From statics, we know:

    $$\left\{\begin{array}{l}\frac{\partial}{\partial t}=0\\\nabla\times E=0\end{array}$$

  \item We also know Poisson's equation:

    $$\nabla^2V=-\frac{\rho}{\varepsilon}$$

  \item In this chapter, we learn:

    $$\nabla\cdot B=0$$
    $$\nabla\times H=J+\frac{\partial}{\partial t}D$$

    \begin{itemize}

      \item Where $\sigma$ is conductivity, we find:

        $$J=\sigma E$$

      \item We can see the $B$ is divergence-less

      \item Furthermore, since we know:

        $$\nabla\times (\text{curl})=0$$

        we can say:

        $$B=\nabla A$$

        since $B=\mu H$:

        $$\mu H=\nabla\times A$$

        this means:

        $$\nabla\times\nabla\times A=\mu J$$

        and finally, since we know $\nabla\times\nabla\times=-\nabla^2+\nabla(\nabla\cdot)$

        $$-\nabla^2A+\nabla(\nabla A)=\mu J$$

        since $A$ is auxiliary, we can say $\nabla\cdot A=0$:

        $$\nabla^2 A=\mu J$$

        Since we know the solution to Poisson's equation, we can apply it here as well, which gives us:

        $$A(r)=\frac{\mu}{4\pi}\int_{\forall}\frac{J(r')}{|r-r'|}\,dr'$$

    \end{itemize}

  \item From Faraday's Law, we can rewrite:

    $$\nabla\times E=-\frac{\partial}{\partial t}\mu H$$

  \item From equations of linear systems, we can obtain:

    $$\nabla\times\tilde{E}(r)=-j\omega\mu\tilde{H}(r)$$
    $$\nabla\times\tilde{H}(r)=j\omega\varepsilon\tilde{E}(r)$$

    \begin{itemize}

      \item The magnetic field phasor would then be:

        $$\tilde{H}(r)=\frac{\nabla\times\tilde{E}}{-j\omega\mu}$$

      \item The electric field phasor would then be:

        $$\nabla\times\tilde{H}=j\omega\varepsilon\tilde{E}$$

    \end{itemize}

\end{itemize}

\end{document}

