%%%%%%%%%%%%%%%%%%%%%%%%%%%%%%%%%%%%%%%%%%%%%%%%%%%%%%%%%%%%%%%%%%%%%%%%%%%%%%%%%%%%%%%%%%%%%%%%%%%%%%%%%%%%%%%%%%%%%%%%%%%%%%%%%%%%%%%%%%%%%%%%%%%%%%%%%%%%%%%%%%%
% Written By Michael Brodskiy
% Class: Fundamentals of Electromagnetics
% Professor: E. Marengo Fuentes
%%%%%%%%%%%%%%%%%%%%%%%%%%%%%%%%%%%%%%%%%%%%%%%%%%%%%%%%%%%%%%%%%%%%%%%%%%%%%%%%%%%%%%%%%%%%%%%%%%%%%%%%%%%%%%%%%%%%%%%%%%%%%%%%%%%%%%%%%%%%%%%%%%%%%%%%%%%%%%%%%%%

\include{Includes.tex}

\title{Pre-Lab Assignment for Experiment 6}
\date{November 8, 2023}
\author{Michael Brodskiy\\ \small Professor: E. Marengo Fuentes}

\begin{document}

\maketitle

\begin{enumerate}

  \item Derive an equation for the critical angle in terms of $n_1$ and $n_2$ when $n_2 < n_1$. Find the critical angle for total internal reflection from water ($n=1.33$) to air ($n=1.00$). 

    Since we know $\theta_c$ occurs when the second angle is $90^{\circ}$, we may write:

      $$n_2\sin(\theta_c)=n_1\longrightarrow \theta_c=\sin^{-1}\left(\frac{n_1}{n_2}\right)$$

    For water to air motion, this becomes:

    $$\theta_c=\sin^{-1}\left( \frac{1}{1.33} \right)$$
    $$\boxed{\theta_c=48.75^{\circ}}$$

\end{enumerate}

\end{document}

